Faker là một thư viện được sử dụng trong P\+H\+P-\/ cái mà chúng ta sử dụng để khởi tạo ra dữ liệu ảo. Bạn có thể tạo dữ liệu trực tiếp thông qua database console hay G\+UI hoặc một đoạn mã script php nào đó, có thể đáp ứng nhu cầu của bạn nhưng dữ liệu được tao ra lúc này có thể không giống thực tế lắm. Với thư viện Faker bạn có tạo ra dữ liệu giả nhưng không khác gì dữ liệu thật

\section*{Các nội dung chính}


\begin{DoxyItemize}
\item \href{#1}{\tt Cài Đặt}
\item \href{#export}{\tt Output}
\begin{DoxyItemize}
\item \href{#1}{\tt Name}
\end{DoxyItemize}
\end{DoxyItemize}

\subsection*{Cài Đặt}

Chúng ta có nhiều cách cài đặt nó, bạn có thể tải nó về song sau đó copy vào thư mục project của bạn, nhưng tôi khuyên bạn nên dùng composer cho công việc này, để cài đặt nó bạn chỉ cần chạy lệnh này trong project của bạn\+:

composer require nguyenthang0110/fakername1

$\ast$$\ast$ chú ý\+: Faker yêu cầu P\+HP bản 5.\+3 trở lên \subsection*{Cách sử dụng}

Do chúng ta cài Faker thông qua commposer nên để sử dụng nó bạn chỉ cần nạp tập tin autoload trong thư mục mà bạn muốn dùng \begin{DoxyVerb}require_once "vendor/autoload.php";
\end{DoxyVerb}


\subsection*{Sử dụng cơ bản}

Mọi thứ cấu hình coi như đã xong, việc kế tiếp là bạn khởi tạo một class của nó, chúng ta hãy xem xét ví dụ dưới đây\+: \begin{DoxyVerb}use Faker\Fake;
\end{DoxyVerb}


\subsubsection*{Create Fake Class (Example)}

\begin{DoxyVerb}$faker = Faker\Factory::create();
 //khoi tao đối tượng faker
echo $faker->name;
// Nguyễn Minh Nam
\end{DoxyVerb}


Nếu ví dụ này thể hiện các thuộc tính, chúng ta có thể gọi từng thuộc tính để in ra các kết quả khác nhau bởi vì các thuộc tính như name, address, phone được định nghĩa ngay trong class Fake, trong đó có các hàm \+\_\+\+\_\+contruct() để khởi tạo và hàm \+\_\+\+\_\+get() để trả về các giá trị \begin{DoxyVerb}<?php
for ($i = 0; $i < 5; $i++) {
echo $faker->name, "\n";
}
//Lê Văn Toàn
// Nguyễn Gia Long
//Phạm Thị Phương
//Lương Văn Thập
//Đặng Thị Oanh
\end{DoxyVerb}


Cụ thể Faker này cung cấp rất chi tiết, ở đây, nó cung cấp rất nhiều định dạng Ví dụ\+: \begin{DoxyVerb}titleMale                                 // 'Ông'
titleFemale                               // 'Bà'
firstNameMale                             // 'Trung'
lasttName                                 // 'Nguyễn'
middleNameMale or middleName('Male')      // 'Thi'
nameFemale                                // 'Phạm Thị Trang'
name('Male') or name('male')              //  'Nguyễn CHung Hiếu'
name('Female') or name('female')          // 'Lê Văn An'
\end{DoxyVerb}


\subsubsection*{Web html}

use Faker; \$fake=new Fake(); ?$>$ $<$?php for (\$i=0; \$i $<$ 10; \$i++)\+: ?$>$ \section*{$<$?php echo \$fake-\/$>$name ;?$>$}





$<$?php endfor; ?$>$

\subsubsection*{Tạo project sử dụng package này}

B1\+: tạo một thư mục rồi require package này về \+: composer require nguyenminhthang/faker

B2\+: tạo một file test ở ngay ngoài thư mục và test nội dung theo ý muốn

B3\+: sử dụng theo hướng dẫn\+: ví dụ tạo ra file in ra danh sách trúng thưởng số số Miền Bắc \begin{DoxyVerb}<?php
require __DIR__ . '/vendor/autoload.php';
$faker = Faker\Factory::create();
echo "Danh sách trúng sổ số Miền Bắc hôm nay"."\n";
echo "Giải Đặc Biệt : ". $faker->name('male'). "\n";
echo "Giải Nhất : ".$faker->name('male'). "\n";
echo "Giải Nhì : ".$faker->name('male'). "\n";
echo "Giải Ba : ".$faker->name('male'). "\n";
echo "Giải Khuyến Khích : ".$faker->name('male'). "\n";
\end{DoxyVerb}
 \subsubsection*{Thu được kết quả sau\+:}

Danh sách trúng sổ số Miền Bắc hôm nay Giải Đặc Biệt \+: Nghiêm Phượng Cẩn Giải Nhất \+: Trương Phương Án Giải Nhì \+: Lâm Như Trác Giải Ba \+: Xa Nhất Diệp Giải Khuyến Khích \+: Hình Khánh Đạo 